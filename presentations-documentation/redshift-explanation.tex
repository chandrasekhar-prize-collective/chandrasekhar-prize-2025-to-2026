\documentclass{article}
\usepackage{graphicx}
\usepackage{amsmath}
\title{Redshift --- an explanation}
\author{Ahaan Shokeen}
\date{2025--12--26}

\begin{document}

\maketitle

Redshift is an increase in the wavelength \(\lambda\) of electromagnetic radiation. This can be defined as a decrease in the frequency \(f\) of electromagnetic radiation, alternatively. The opposite phenomenon is blueshift: a decrease in \(\lambda\) and an increase in \(f\).\\
There are 3 forms of redshift in astronomy and cosmology:
\begin{itemize}
    \item Doppler redshift
    \item Gravitational redshift
    \item Cosmological redshift
\end{itemize}

The value of a colour shift; i.e., its offset from the original colour, is denoted by \(z\), which is dimensionless. \(z\) is positive for redshifts, and negative for blueshifts.\\
To calculate redshift, the following formulae exist.
\\
Using wavelength:
\[
z = \frac{\lambda_{\text{observed}}-\lambda_{\text{emitted}}}
{\lambda_{\text{emitted}}}
\]
Using frequency:
\[
z = \frac{f_{\text{emitted}}-f_{\text{observed}}}
{f_{\text{observed}}}
\]


\end{document}