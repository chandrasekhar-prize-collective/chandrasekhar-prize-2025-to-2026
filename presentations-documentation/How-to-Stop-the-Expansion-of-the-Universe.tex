\documentclass{article}
\usepackage{amsmath}
\usepackage{graphicx} % Required for inserting images

\title{How to Visually Enhance Redshift Interpretation---a Theoretical Approach}
\author{Daksh Jain}
\date{January 2026}

\begin{document}

\maketitle
\begin{center}
    \section*{Formulae and Derivation}
\end{center}
Cosmological redshift revolves around one primary formula: \\

\begin{equation*}
     \boxed{1+z=\frac{\lambda_{observed}}{\lambda_{rest}}}
\end{equation*} \\

But to understand this, we must go into the core of solid mechanics. Which uses: \\

\begin{equation*}
    \text{Strain}=\frac{\Delta L}{L_0}
\end{equation*} \\

Where you stretch a rod, measure how much longer it got, and divide it by the original length. \\

Now when we apply this to redshift---it sounds familiar; because redshift is strain in space. 
Going to the bare definition of redshift, it is the stretching of light as it travels through the expanding universe. Intrinsically, the change we observe is the change in wavelength because of this stretching. Now, the two terms that change behave similarly to strain. \\

One is $\lambda_{rest}$ or $\lambda_{emit}$ which refer to the original wavelength emitted by the object—caused by electrons (or charged particles) releasing packets of energy (in the form of electromagnetic radiation) when dropping energy levels. Whereas, $\lambda_{observed}$ refers to the wavelength observed after stretching through the universe. \\

Now we can observe a pattern, and deduce: \\

\begin{equation*}
    z = \frac{\Delta \lambda}{\lambda_0}
\end{equation*} \\

Which can be rewritten as: \\

\begin{equation*}
    z = \frac{\lambda_{observed}-\lambda_{rest}}{\lambda_{rest}}
\end{equation*} \\

Which can be further simplified into: \\

\begin{equation*}
    z = \frac{\lambda_{rest}(\frac{\lambda_{observed}}{\lambda_{rest}}-1)}{\lambda_{rest}}
\end{equation*} \\

Cancel out. \\

\begin{equation*}
    z = \frac{\lambda_{observed}}{\lambda_{rest}}-1
\end{equation*} \\

And we have finally reached the explicit form. Now to derive it to capture the original rest-frame wavelength. \\

\begin{equation*}
    \lambda_{rest} = \frac{\lambda_{observed}}{z + 1}
\end{equation*} \\

However, to know this, we must know the redshift itself. Looks like we're in a loop! To fix this issue, we require more than just a telescope. Though it may \textit{look} like we hit a wall, a bit more information combined with extra hardware can give us our result.
Atoms emit spectral lines that at very specific wavelengths corresponding to precise energy levels. When an electron drops energy levels, it emits light---whereas when it increases, it absorbs. Emission lines are bright lines emitted by hot, thin gas and absorption lines are dark lines emitted by a cool gas. Each element has a unique  pattern that do shift with expansion; however, ratios and patterns remain identical. This can be used to our advantage. With a spectrograph, we can capture these spectral lines, and compare them to existing databases to identify how they would originally look in the rest-frame. Now that we have this information, we can use the previous formulae to get the enhanced interpretation. \\

Now let's see this in action. For Jupiter, a typical spectrograph would look similar to this: \textit{See: figure 1} \\

\begin{figure}
    \centering
    \includegraphics[width=0.75\linewidth]{JUPTER.png}
    \caption{(Encrenaz et al., 2022, Figure 2).}
\end{figure} \\

Now, our algorithm can compare this to other spectrographs of Jupiter. Once a match is found, it is easy to obtain the rest-frames. First, identify the dense regions of that particular element in the object, then map out each region individually. This is known as \textbf{spectral mapping} and may take time, but will ensure accuracy. 

\end{document}
